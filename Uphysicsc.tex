\documentclass[aps,prl,reprint]{revtex4-1}
\makeatletter
\def\Dated@name{}
\makeatother

\usepackage[utf8]{inputenc}
\usepackage[english]{babel}
\usepackage{amsmath}
\usepackage{amsthm}
\usepackage{amssymb,units}
\usepackage{graphicx}
\usepackage{dcolumn}
\usepackage{float, fancyhdr}
\usepackage{footnote, minted}


\pagestyle{fancy}
\fancyhf{}
\rhead{Team number: 891}
\begin{document}

	
	%Título do Relatório
	
	\author{Team number: 891\\ Problem A\\Aquí va el títol}
	
	\begin{abstract}
		
		\begin{center}
			
			\rule{15cm}{1pt} \\
			
		\end{center}
		
		Ací va l'abstract que queda bonic blablablablablablablablablablablabla
		\begin{center}
			
			\rule{15cm}{1pt} \\
			
		\end{center}
		
	\end{abstract}
	
	\maketitle
	\onecolumngrid\newpage\twocolumngrid
	\section{Introducció}
	+ blablablablablalbalbalblalbalblala
	
	\begin{minted}{python}
	import numpy as np
	
	def incmatrix(genl1,genl2):
	m = len(genl1)
	n = len(genl2)
	M = None #to become the incidence matrix
	VT = np.zeros((n*m,1), int)  #dummy variable
	
	#compute the bitwise xor matrix
	M1 = bitxormatrix(genl1)
	M2 = np.triu(bitxormatrix(genl2),1) 
	
	for i in range(m-1):
	for j in range(i+1, m):
	[r,c] = np.where(M2 == M1[i,j])
	for k in range(len(r)):
	VT[(i)*n + r[k]] = 1;
	VT[(i)*n + c[k]] = 1;
	VT[(j)*n + r[k]] = 1;
	VT[(j)*n + c[k]] = 1;
	
	if M is None:
	M = np.copy(VT)
	else:
	M = np.concatenate((M, VT), 1)
	
	VT = np.zeros((n*m,1), int)
	
	return M
	\end{minted}
	\section{References}
	[1] J. COSTA QUINTANA. En principi és a dir. $5{\textsuperscript{\underline{a}}}$ edição. São Paulo - SP: Edgard Blucher, 2014. 
	
	
	
	
	
	
\end{document}